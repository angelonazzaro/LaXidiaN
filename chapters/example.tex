\chapter{Example}

\section{Section 1}
Lua\TeX{} is a \textit{toolkit}—it contains \textbf{sophisticated} software tools and components with which you can construct (typeset) a wide range of documents. The sub-title of this article also poses two questions about Lua\TeX: What is it—and what makes it so different? The answer to “What is it?” may seem obvious: “It’s a \TeX{} typesetting engine!” Indeed it is, but a broader view, and one to which this author subscribes, is that Lua\TeX{} is an extremely versatile \TeX-based document construction and engineering system.

\subsection{Subsection}
The goal of this first article on Lua\TeX{} is to offer a context for understanding what this TeX engine provides and why/how its design enables users to build/design/create a wide range of solutions to complex typesetting and design problems—perhaps also offering some degree of “future proofing” 

\subsubsection{SubSubSection}
\lipsum[1]

\infobox{Info box}{
    \lipsum[1]
}{info:box}

\infobox{Info box}{
    \lipsum[1]
}{}

\tipbox{Tip box}{
    \lipsum[1]
}{tip:box}

\tipbox{Tip box}{
    \lipsum[1]
}{}

\faqbox{FAQ box}{
    \lipsum[1]
}{faq:box}

\faqbox{FAQ box}{
    \lipsum[1]
}{}

\successbox{Success box}{
    \lipsum[1]
}{suc:box}

\successbox{Success box}{
    \lipsum[1]
}{}

\errorbox{Error box}{
    \lipsum[1]
}{err:box}

\errorbox{Error box}{
    \lipsum[1]
}{}

This text is referred to box \ref{info:box}. \\
This text is referred to box \ref{tip:box}. \\
This text is referred to box \ref{faq:box}. \\
This text is referred to box \ref{suc:box}. \\
This text is referred to box \ref{err:box}. \\
This text is referred to box \ref{cus:box}. \\

\section{Title}


\definecolor{customcolor}{HTML}{C6E2FF}
\custombox{customcolor}{\faCodeFork \space \space Custom box}{\lipsum[1]}{cus:box}
\custombox{customcolor}{\faCodeFork \space \space Custom box}{\lipsum[1]}{}

\quotebox{\lipsum[1]}{}
\quotebox{\lipsum[1]}{quote:box}

\theorembox{Nysquits}{Nyquist, noto anche come il teorema di campionamento di Nyquist, stabilisce le condizioni necessarie per la campionatura di un segnale analogico in modo che possa essere ricostruito senza perdita di informazioni. $$f_s \geq 2 \cdot f_{\text{max}}$$}{theorem:nysquits}
\theorembox{}{\lipsum[1]}{theorem:case}

\proofbox{\lipsum[1]}{theorem:nysquits}{proof:case}

\chapter{Example 2}
\lipsum[1]

$$ \sum_{i = 0} a^{i} = 0 $$ 


\begin{equation} \large
H(y, \hat{y}) = -\sum_{i} (y_i \log(\hat{y}_i) + (1 - y_i) \log(1 - \hat{y}_i))
\end{equation}